\documentclass[twocolumn,prl]{revtex4-1}
%\documentclass{article}
\begin{document}
\title{Capillary Attraction of }
\author{Student A}
\author{Student B}
\author{Student C}
\author{Professor}
\affiliation{Brown University}

\begin{abstract}
	Lorem ipsum dolor sit amet, consectetur adipisicing elit, sed do eiusmod tempor incididunt ut labore et dolore magna aliqua. Ut enim ad minim veniam, quis nostrud exercitation ullamco laboris nisi ut aliquip ex ea commodo consequat. Duis aute irure dolor in reprehenderit in voluptate velit esse cillum dolore eu fugiat nulla pariatur. Excepteur sint occaecat cupidatat non proident, sunt in culpa qui officia deserunt mollit anim id est laborum.
\end{abstract}

\maketitle

\section{Introduction}
	
	Thin floating objects mutually attract through capillary interactions in a shape-dependent mechanism. These capillary interactions, commonly known as the “Cheerios Effect”, is a gravity and surface wetting (any better terms…?) driven process communicated along the interface on which these object float.  This attractive mechanism has been observed or postulated in locomotion of aquatic plant seeds, directed colloidal self-assembly, .... Thus far, studies has been focused on smooth objects and there is relatively less attention on objects with sharp corners, which we refer to as sharp objects. 
	
	In general, sharp objects behave differently from smooth objects because the corners pins the local interface or contact line onto a discontinuity. Figure 1 presents initial qualitative experiments of the attractive behavior of two floating 1'' wide acrylic triangles as they approach each other, viewed from above. The triangles rotate such that corners kiss first, after which they will fold or slide into a more stable final configuration. Prior to kissing, it appears that attraction is localized at the corners. This effect is a poorly understood consequence of the discontinuous pinned contact line. For floating spheres and cylinders, there are analytical solutions using multipole expansions in analogy to electrostatic to show exponential decay of attraction. We present an alternative perspective, however, using conformal mapping to show a power-law scaling near a corner which extends further than an exponential.

//Paragraph on experimental setup, boundary integral method//
	 

\section{Methods}
//Explanation of measuring surface gradients to calculate force - explain the equation to calculate force//

We calculate the attractive force between two floating objects by measuring the surface gradients of the deformed interface ... DIC

\begin{equation}
F_h \approx \frac{\sigma}{2} \int _C [\frac{u^2}{l_c^2}- u_n^2 +u_t^2] dt
\end{equation}

Figure 2 shows that calculated force between two 1/16" thick acrylic circles as a function of distance.
//Show experimental set-up//

//As a check, show attractive force between spheres and compare with literature//

Force from deformation of meniscus. Measure Meniscus deformations - Verify using circles. Use for triangles for d ~ lc and $d << lc$.

\section{Results}
//Show derivation for conformal mapping on angles//

//Show ux scaling for the inner and outer region//

//Show Force plot and regions of scaling//

\section{Discussion}
What does this mean?

Force concentrated at vertices is much stronger than force between smooth edges.

\section{Figures}
Figure 1: Analog of introduction in figures

Figure 2: Schematic for experiment and representative results 

Figure 3: Power law scaling and force vs. distances.

Figure 4: On the boundary integral methods

\section{notes}



\end{document}
